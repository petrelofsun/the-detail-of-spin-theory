\documentclass[a4paper]{article}
\usepackage{amsmath, amssymb, amsthm}
\usepackage{fancyhdr}
\usepackage{enumitem}
\usepackage{physics}
\usepackage[english]{babel}
\usepackage{geometry}
\usepackage{graphicx}
\usepackage{hyperref}
\usepackage{mathrsfs}
\usepackage{epstopdf}
\usepackage{float}
\usepackage{bbm}
\usepackage{cases}

\title{\textbf{Lorentz Boost of Spin Vector}}
\author{Lingyang Kong}
\date{}

\begin{document}
\maketitle
\par Arbitrary direction Lorentz boost of the exponential form of the generating elements,
\begin{equation}
    \Lambda(y\hat{\boldsymbol{p}})=exp(-iy\hat{\boldsymbol{p}}\cdot\boldsymbol{K})
\end{equation}
\par In which $y$ is rapidity of moumentum direction And $\hat{\boldsymbol{p}}=(sin\theta cos\phi, sin\theta sin\phi, cos\theta)$.
\par For left-handed Weyl spinor, $K_{i}=i\sigma_{i}/2$, 
\begin{equation}
    \begin{split}
            \Lambda(y\hat{\boldsymbol{p}})_{L}
            &=\sum_{n}{\frac{1}{n!}(\frac{y\hat{\boldsymbol{p}}\cdot \boldsymbol{\sigma}}{2})^n}\\
            &=\sum_{m}{\frac{1}{(2m)!}(\frac{y}{2})^{2m}(\hat{\boldsymbol{p}}\cdot \boldsymbol{\sigma})^{2m}}
            +\hat{\boldsymbol{p}}\cdot \boldsymbol{\sigma}\sum_{m}{\frac{1}{(2m+1)!}(\frac{y}{2})^{2m+1}(\hat{\boldsymbol{p}}\cdot \boldsymbol{\sigma})^{2m}}\\
    \end{split}
\end{equation}
\par We know that $(\hat{\boldsymbol{p}}\cdot \boldsymbol{\sigma})^2=1$. And $cosh(x)=\sum{\frac{1}{(2m)!}x^{2m}}$, $sinh(x)=\sum{\frac{1}{(2m+1)!}x^{2m+1}}$ according to Taylor expansion. We can get,
\begin{equation}
    \Lambda(y\hat{\boldsymbol{p}})_{L}=cosh(\frac{y}{2})+sinh(\frac{y}{2})\hat{\boldsymbol{p}}\cdot \boldsymbol{\sigma}
\end{equation}
\par For right-handed Weyl spinor, $K_{i}=-i\sigma_{i}/2$. similarly,
\begin{equation}
    \Lambda(y\hat{\boldsymbol{p}})_{R}=cosh(\frac{y}{2})-sinh(\frac{y}{2})\hat{\boldsymbol{p}}\cdot \boldsymbol{\sigma}
\end{equation}
\par In which $y=ln\frac{E+p}{m}$, substitute into Eq.(3)(4), we can get
\begin{equation}
    \Lambda(y\hat{\boldsymbol{p}})_{L}=\sqrt{\frac{E+m}{2m}}+\sqrt{\frac{1}{2m(E+m)}}\boldsymbol{p}\cdot\boldsymbol{\sigma}
\end{equation}
\begin{equation}
    \Lambda(y\hat{\boldsymbol{p}})_{R}=\sqrt{\frac{E+m}{2m}}-\sqrt{\frac{1}{2m(E+m)}}\boldsymbol{p}\cdot\boldsymbol{\sigma}
\end{equation}
\par And we can get,
\begin{equation}
    \Lambda(y\hat{\boldsymbol{p}})_{L}\cdot \Lambda(y\hat{\boldsymbol{p}})_{R}=\Lambda(y\hat{\boldsymbol{p}})_{R}\cdot\Lambda(y\hat{\boldsymbol{p}})_{L}=\mathbbm{1}
\end{equation}
\par i.e.
\begin{equation}
       \Lambda(y\hat{\boldsymbol{p}})_{L}^{-1}=\Lambda(y\hat{\boldsymbol{p}})_{R}\\
\end{equation}
\par Because $\sigma^{\dag}_{i}=\sigma_{i}$,
\begin{equation}
    \Lambda(y\hat{\boldsymbol{p}})_{L}^{\dag}=\Lambda(y\hat{\boldsymbol{p}})_{L}
\end{equation}
\par For Dirac spinor $\psi=\begin{pmatrix}
    \psi_{L}\\
    \psi_{R}
\end{pmatrix}$, Lorentz boost matrix,
\begin{equation}
     \Lambda(y\hat{\boldsymbol{p}})=\begin{pmatrix}
          \Lambda_{L} & 0\\
          0 & \Lambda_{R}
     \end{pmatrix}
\end{equation}
\par And,
\begin{equation}
     \Lambda(y\hat{\boldsymbol{p}})^{\dag}=\begin{pmatrix}
          \Lambda_{L} & 0\\
          0 & \Lambda_{R}
     \end{pmatrix}
\end{equation}
\par For $s^{*\mu}\gamma_{\mu}$, in which $s^{*\mu}=s(0,\vec{n})$ is spin vector in rest frame, perfrom a Lorentz boost, we can get,
\begin{equation}
        s^{*\mu}\gamma_{\mu} 
        \to \Lambda(y\hat{\boldsymbol{p}})s^{*\mu}\gamma_{\mu}\gamma_{0}\Lambda(y\hat{\boldsymbol{p}})^{\dag}\gamma_{0}=s^{*\mu}\Lambda(y\hat{\boldsymbol{p}})\gamma_{\mu}\gamma_{0}\Lambda(y\hat{\boldsymbol{p}})^{\dag}\gamma_{0}
\end{equation}
\par Assume $s^{\mu}\gamma_{\mu}=s^{*\mu}\Lambda(y\hat{\boldsymbol{p}})\gamma_{\mu}\gamma_{0}\Lambda(y\hat{\boldsymbol{p}})\gamma_{0}$, in which $s^{\mu}=s(s^{0},\vec{s})$ $\gamma^{\mu}$ takes the Weyl representation, we can get,
\begin{equation}
\begin{split}
s^{\mu}\gamma_{\mu}
&=s^{*\mu}\Lambda(y\hat{\boldsymbol{p}})\gamma_{\mu}\Lambda(y\hat{\boldsymbol{p}})\\
&=s^{*\mu}
\begin{pmatrix}
\Lambda_{L} & 0\\
0 & \Lambda_{R}
\end{pmatrix}
\begin{pmatrix}
    0 & \sigma_{\mu}\\
    \bar{\sigma}_{\mu} & 0
\end{pmatrix}
\begin{pmatrix}
\Lambda_{R} & 0\\
0 & \Lambda_{L}
\end{pmatrix}\\
&=s^{*\mu}
\begin{pmatrix}
    0 & \Lambda_{L}\sigma_{\mu}\Lambda_{L}\\
    \Lambda_{R}\bar{\sigma}_{\mu}\Lambda_{R} & 0
\end{pmatrix}
\end{split}
\end{equation}
\par i.e.
\begin{numcases}{}
    s^{\mu}\sigma_{\mu}=s^{*\mu}\Lambda_{L}\sigma_{\mu}\Lambda_{L}\\
    s^{\mu}\bar{\sigma}_{\mu}=s^{*\mu}\Lambda_{R}\bar{\sigma}_{\mu}\Lambda_{R}
\end{numcases}
\par Expanding the equations,
\begin{gather}
        s^{0}-\boldsymbol{s}\cdot\boldsymbol{\sigma}=-s\frac{E+m}{2m}(\boldsymbol{n}\cdot\boldsymbol{\sigma})-s\frac{1}{2m}\left\{(\boldsymbol{p}\cdot\boldsymbol{\sigma}),(\boldsymbol{n}\cdot\boldsymbol{\sigma})\right\}-s\frac{1}{2m(E+m)}(\boldsymbol{p}\cdot\boldsymbol{\sigma})(\boldsymbol{n}\cdot\boldsymbol{\sigma})(\boldsymbol{p}\cdot\boldsymbol{\sigma})\\
        s^{0}+\boldsymbol{s}\cdot\boldsymbol{\sigma}=s\frac{E+m}{2m}(\boldsymbol{n}\cdot\boldsymbol{\sigma})-s\frac{1}{2m}\left\{(\boldsymbol{p}\cdot\boldsymbol{\sigma}),(\boldsymbol{n}\cdot\boldsymbol{\sigma})\right\}+s\frac{1}{2m(E+m)}(\boldsymbol{p}\cdot\boldsymbol{\sigma})(\boldsymbol{n}\cdot\boldsymbol{\sigma})(\boldsymbol{p}\cdot\boldsymbol{\sigma})
\end{gather}
\par We know that $(\boldsymbol{a}\cdot\boldsymbol{\sigma})(\boldsymbol{b}\cdot\boldsymbol{\sigma})=(\boldsymbol{a}\cdot\boldsymbol{b})\mathbbm{1}+i(\boldsymbol{a}\times\boldsymbol{b})\cdot\boldsymbol{\sigma}$, 
\begin{gather}
    \left\{(\boldsymbol{p}\cdot\boldsymbol{\sigma}),(\boldsymbol{n}\cdot\boldsymbol{\sigma})\right\}=2(\boldsymbol{p}\cdot\boldsymbol{n})\mathbbm{1} \\
    (\boldsymbol{p}\cdot\boldsymbol{\sigma})(\boldsymbol{n}\cdot\boldsymbol{\sigma})(\boldsymbol{p}\cdot\boldsymbol{\sigma})=2(\boldsymbol{p}\cdot\boldsymbol{n})(\boldsymbol{p}\cdot\boldsymbol{\sigma})-\left|\boldsymbol{p}\right|^2(\boldsymbol{n}\cdot\boldsymbol{\sigma})
\end{gather}
\par Substitute in Eq.(16)(17), we can get, 
\begin{equation}
     s^{0}-\boldsymbol{s}\cdot\boldsymbol{\sigma}=-s\frac{\boldsymbol{p}\cdot\boldsymbol{n}}{m}-s[\boldsymbol{n}+\frac{1}{m(E+m)}(\boldsymbol{p}\cdot\boldsymbol{n})\boldsymbol{p}]\cdot\boldsymbol{\sigma}
\end{equation}
\par Finally, we conclude that the spin vector after Lorentz boost transformation, 
\begin{equation}
    s^{\mu}=s(-\frac{\boldsymbol{p}\cdot\boldsymbol{n}}{m},\boldsymbol{n}+\frac{1}{m(E+m)}(\boldsymbol{p}\cdot\boldsymbol{n})\boldsymbol{p} \ )
\end{equation}
\end{document}
